\documentclass{article}
\usepackage{graphics}

\usepackage{hyperref}
\usepackage{fancyhdr}


\voffset = -20 mm
\textheight = 240 mm
\hoffset = -8 mm
\textwidth = 152 mm
\footskip = 18 mm



\begin{document}

\section*{uFEM User's Guide}

\tableofcontents

\section{Introduction}

\subsection{Overview}

 The uFEM is a computational tool for a static finite
 element analysis of structures. It includes a command-line
 solver and a command-line and graphics integrated pre-
 and postprocessor.



\subsection{Name}

 The FEM stands for a "Finite Element Method". The starting
 "u" initialy meant a Greek letter "mu" (because of the small
 size of the code) but now it has no specific meaning.
 There were attempts to translate it as an "ultimate FEM" but 
 much
 more realistic meaning of the "u" is an "unfinished".

 In this text a "uFEM" refers to the whole package, while the solver	
 is called a "fem" and user interfaces are  a"tfem" (the terminal
 one) and a "gfem" (the graphical one).



\subsection{Current possibilities}

\subsubsection{Finite elements}
\begin{itemize}
\item  2D trusses (element 1) and 3D trusses (12)
\item  2D beams (3)
\item  plane strain and plane stress elements (2, 11)
\item  3D elements (hexahedron 4, bricks 8, 9, 10)
\item  2D slab (5)
\end{itemize}

\subsubsection{Material models}
\begin{itemize}
\item  linear elastic material
\item  elastoplastic (Chen) with Ohtani's hardening
\item  smeared crack model for concrete (for 2D only)
\item  several experimental material types
\end{itemize}

\subsubsection{Analysis types}
\begin{itemize}
\item  linear solution (obviously...)
\item  linear solution with one-directional boundary conditions
\item  non-linear Newton-Rapshon solution (for physical non-linearity)
\item  non-linear Arc-Length Method solution (for physical non-linearity)
\end{itemize}

\subsection{uFEM components}

\subsubsection{Solver}

 The solver is a non-interactive command-line based application.
 It can be controlled by input files and through command-line
 parameters. The solver is a highly portable and can run on
 all current major operating systems and hardware platforms.



\subsubsection{Interactive pre and post-processor}

 There is an integrated preprocessing and postprocessing 
 environment. However, it its possibilities are relatively limited.
 It is basicaly command-driven and there is a terminal-only
 version and a graphical (GUI) application. The GUI one
 offers menus and dialogs for many commands.
 The preprocessor has some limited capabilities of structured
 mesh generation (2D and 3D).



\section{Solver}

\subsection{Invocation and running}

 The uFEM solver is started by "fem [parameters]". 



\subsubsection{Basic command-line [parameters]}
\begin{itemize}
\item  -p ... indicates that input file is preparsed; uFEM-generated files (by the "export" command) are but hand-written file are not.
\item  -i FILE ... name of an input file; to use a standard input, use the "-si" instead
\item  -o FILE ... name of an output file; to use a standard output, use the "-so" instead
\end{itemize}

\subsubsection{Linear solution}

 No additional parameters are necessary for a linear solution.



\subsubsection{One-direction boundary conditions}

 The one-directional boundary conditions must be defined
 in the model. Names of one-directional boundary conditions 
 the "-" (for compression-only) or the "+" (for tension only) symbols
 (examples: "ux+" or "stiffy-").

 The iterative solution is currently possible  for linear
 solver only.



\paragraph{Required command line parameter:}
\begin{itemize}
\item  -d ... starts iterative solver for one-directional boundary conditions.
\end{itemize}

\subsubsection{Non-linear solution}
\begin{itemize}
\item  -nls TYPE ... type of non-linear solver (0 and 1 are for a linear solution, 2 is for a Newton-Raphson)
\item  -nlstp NUMBER ... number of steps of the solution
\item  -nlf FILE ... file to save convergence data
\item  -ln NODE ... number of node that will be monitored (load level and sx, sy, sxy stresses will be saved)
\item  -lf FILE name of file to store monitoring data
\item  -ose N ... save every N-th substep
\item  -osf FILE ... template for names of substeps
\end{itemize}

\subsubsection{Multi-step solution}

 Unlike many other finite element systems the uFEM doesn't
 support multi-step solutions (e. g. modelling of construction
 process of building or time-dependent problems) out of the box.
 
 But it is possible to read the results of the previous solution
 from a file or from a pipe. And this is a way that uFEM use.
 
 These commands can be used to read data from previous step:


\begin{itemize}
\item  -r RFILE ... read result data from the file named RFILE
\item  -sr ... read result data from the standard input
\end{itemize}

\subsection{Input file format}

 The format is described in the feminput(5) manual page.



\subsection{Output file format}

 TODO



\section{User Interface}

\subsection{Introduction	}

 User interface can be used to create a computational model,
 to modify it (a preprocessor), to call a solver
 and to review results (a postprocessor).
 There are two alternatives of the user interface -- text-only
 terminal version and a graphics (GUI) one. They are functionally
 equivalent, except the graphics capabilities (visualisations of
 data, visual selections,...) are not available in the text
 version.



\subsection{Basic principles}

\subsubsection{Input conventions}

 The program is command-driven and all GUI operations are
 actually only front-ends to these commands. It is important
 to remember that a comma "," symbol is a command parameter
 separator and thus it should not be used in any GUI dialog!
 The decimal separator is always the dot ".". Even if other
 applications use commas.

 Many of numeric parameters (node numbers, coordinates) can
 be ommited. In this case the zero (0.0) is used
 for floating-point values but the value highest than previous
 maximum value is used for identifiers (so it is save to don't
 give a numbers of nodes when they are created, for example).
 If "yes" or "no" input is expected an empty value means "no". 

 For floating point entries (only!) the mathematical
 expression evaluation can be used. So it can be written
 "5*5" instead of "25". Only the "+". "-", "*" and "/" and 
  operators are possible. No variables nor functions
 nor constants and variables can be used inside expressions.



\subsubsection{Work organisation}

 The uFEM requires that hings must be difined in this order:


\begin{enumerate}
\item  finite element types
\item  real set data (e.g. areas of beams, thicknesses of walls etc.
\item  material types
\item  geometric model - lines, areas, volumes (optional)
\item  finite elements
\item  loads and boundary conditions
\end{enumerate}

 Please note that you can create finite element model straightly
 from nodes and elements or through a geometric model and a mesh
 generator. But the both alternatives require the element type
 data, real set data and material data must be given first.

 There is one important limitation: once defined, material nor
 element type can not be changed (but values of their properties
 can be).


\begin{itemize}
\item  
\end{itemize}

\subsection{Invocation and basic setup}

 Just call "tfem" for a text-only version or "gfem" for a GUI one.

 A succesful usage needs some configuration. There is 
 a configuration file that can be stored in these
 locations (checked in a given order):


\begin{itemize}
\item  /etc/tgfemrc (or C:/Program Files/ufem/tgfemrc.txt if applicable)
\item  ~/.tgfemrc
\item  in a file named according to a UFEM\_CONFIG system variable (if any)
\end{itemize}

 Please note that system variables may be unaccessible in
 some environments.
 The file includes normal uFEM commands. The essential ones are:


\begin{itemize}
\item  "setsolver,full\_path\_to\_solver"
\item  "datadir,directory\_name" (place where automaticaly named files are stored)
\item  "ggeom,x0,y0,width,height" (program window position and size)
\item  "plotprop,autoreplot,yes" (automatic refresh of plotted data - comfortable but slow)
\item  "outauto,yes" (text output - listings etc. will be saved to automatically named files)
\item  "outview,viewer\_program\_name" (viewer for text outputs)
\end{itemize}

 The viewer can be any program which is able to show text
 files and which can receive files to see through command
 line. Most text editor and www browser can be used here.
 The ouput can be a plain text file (can be specified
 through "outform, plain" command) or a HTML file ("outform,
 html") or a LaTeX file (outform, tex). The "outform" command
 can be used in configuration file or can be called in any
 moment during the program use.



\subsection{General notes for user interface}

\subsubsection{Outputs}

 Most of text outputs are printed to the Terminal window (in most
 cases it is a black window with text outputs which is started
 in the same time as the uFEM's graphical window).

 Outputs from commands which end with "list" (or form
 commands selected through "List" program menu) can be
 printed or to Terminal window or they can be shown in
 separate viewer windows (the viewer program is specified
 through "outview" command in the configuration file -- see
 above in the "Invocation and basic setup" section).



\subsubsection{Help}

 There is a help function: just use the "help" command.
 The help will be printed to the terminal.

 To obtain help about available finite element types
 and the parameters of related real sets you can 
 use the "help,e,type\_number". The "type\_number" parameter
 can be number or "all".

 To obtain help about available material types
 and their parameters you can 
 use the "help,e,type\_number". The "type\_number" parameter
 can be number or "all".



\subsubsection{Selections}

 Things (nodes, elements, supports, loads, keypoints
 and geometric entities) can be selected or unselected.
 By default everything is selected but many commands operate
 with whole data sets (e.g. with all nodes) so in some cases
 it is needed to have only part of data to be selected.
 Unselected data can not be changed by any command (except
 the "cleandata" that deletes everything).

 The selection command has a form: "*sel,style,type,direction,value1,value2".
 The most important commands are:


\begin{itemize}
\item  nsel .. selects nodes
\item  esel .. selects elements
\item  dsel .. selects supports (boundary conditions)
\item  fsel .. selects loads on nodes (forces)
\item  ksel .. selects keypoints
\item  gesel .. selects geometric entities
\end{itemize}

 The "style" parameter can be:


\begin{itemize}
\item  s .. select from full set
\item  r .. select from current selection
\item  add .. add to current selection
\item  u .. unselect from current selection
\item  all .. select all (no following parameters are needed)
\item  inve .. inverse current selection (no following parameters are needed)
\end{itemize}


 The "type" depends on a command but common values are:


\begin{itemize}
\item  id .. item identifier ("par1" and "par2" means from..to here)
\item  pos .. position (nodes, keypoints, elements only), "x", "y" or "z" position must be specified
\item  mat, elem, rs .. elements can be selected through identifiers of their material and element types or real sets.
\end{itemize}

\subsection{Opening and saving of model}

 Existing model can be opened by
 "resume,full\_path\_and\_name\_of\_file" command. A recommended
 extension for a model file is ".db" (must be included in the
 "full\_path\_and\_name\_of\_file" parameter).

 Created or edited model can be saved by
 "save,full\_path\_and\_name\_of\_file" command. If the "save" command
 is called without parameter then default name (previosly defined
 by "jobname,my\_name" command") will be used. The same applies to
 "resume" command if called without parameter.

 There is also a possibility to import a model from a solver's input
 file (it can be used if an original *.db file is damaged or
 unavailable). It is accessible through "import,fem,file\_name"
 command.



\subsection{Preprocessing}

\subsubsection{Initialization}

 The preprocessor is started by the "prep" command. It is
 usually not necessary to call it because the preprocessing
 is a default initial mode. But it is necessary after return
 from postprocessor - outside the preprocessor, it is impossible
 to modify the computational model. Also calling of any
 command that is not allowed in the postprocessor will
 cause switch to the preprocessor mode.



\subsubsection{Element type}

 The element type is selected by the: "et,identitier,type\_number".
 The type\_number is 1 for 2D link etc. 

 To delete an element type use: "etdel,identifier". Please note
 that it is impossible to delete an element type that is already
 used in any existing finite element in the model.

 A summary of defined element types can be shovn by: "etlist".



\subsubsection{Real constants set}

 A finite element properties (sans node locations) are refered
 here as "real constants". They include thicknesses of 2D
 elements, cross-section properties of beams etc.

 To define a real set the "rs,id,type\_number" command should be used.
 The new real set will be numbered "id" and will of the "type\_number"
 element type (it is the same as fot the "et" command).

 A summary of defined real constants and their
 values can be shown using a command "rslist".



\subsubsection{Material type}

 A material type can be defined throught the "mat,id,type\_number"
 command (use 1 to define a Hook-type material).

 The values can be set with the "mp,type,material\_id,value"
 command.

 A summary of defined mmaterial types and their properties
 can be shown using a command "mplist".



\subsubsection{Geometric model}

\paragraph{Introduction}

 Geometric model represents the shape of a modelled
 structure. It serves as a basis for a creation
 of a finite element model. Loads and supports can not
 be appplied to a geometric model.

 It includes key points and geometric entities (lines,
 areas, volumes). Key points are used for definition of
 geometric entities and geometric entities serve for
 creation of finite element mesh.



\paragraph{Key points}

 Individual key points can be created through "k,number,x,y,z"
 command.

 Key points can be copied with the "kgen,number,offset\_x,offset\_y,offset\_z" command.
 Note that the "number" parameter means number of newly created
 key points (excluding the original ones). All existing key points
 are copied, if only a subset of key points is meant to be copied
 then these nodes must be selected first.

 The "kdel,node\_number" can be used to delete a key point.



\paragraph{Geometric entities}

 Available geometric entities include lines (defined by 2 keypoints), areas
 (defined by 4 keypoints) and volumes (defined by 8 keypoints).
 The command for creation of different types of entities are giwen below
 but there is also a simpler command for creation of entity of predefined
 size and position in space.
 The command is "gesize,identitier,type,x0,y0,z0,length\_x,length\_y,length\_z"
 The type should be:


\begin{itemize}
\item  1 ... line
\item  2 ... rectangle
\item  3 ... brick (with 8 keypoints)
\item  4 ... curvilinear brick (with 20 keypoints)
\end{itemize}

 The x0,y0,z0 define the position of left bottom angle of entity (on a brick
 it is on its back side).
 The "length\_i" parameters define dimensions of entity.

 The dimensions and position of existing  entities can be edited through modifications
 of coordinates of their keypoints.



\subparagraph{Properties	}

 Default properties (before the actual antities are created) can be set
 by "edef,element\_type,real\_set,material" command.
 The parameters are a user-defined numbers of properties which
 were previously defined by "et", "rs" and "mat" commands.
 
 Properties of existing entity can be changed with use of
 "gep,entity\_number,element\_type,real\_set,material" command.



\subparagraph{Edge division}

 Geometry entities are mainly used as bases for creation of finite element meshes.
 Only regular meshes can be greated by now and these meshes require defined divisions
 of edges of entities. The default division (before the entities are created) can 
 be set be "ddiv,divivision\_number" command. The "divivision\_number" parameter
 is a integer value.

 Edge divisions of existing entity can be set by 
 "gediv,entity\_identifier,division\_x,division\_y,division\_z" command.



\subparagraph{Lines}

 Individual lines can be created with the "l,number,keypoint1,keypoint2" command.
 The "number" means number of newly created line or it can be a number of existing one 
 (in this case the existing line is changed to new values). 

 The "ldel,node\_number" can be used to delete a line.



\subparagraph{Areas}

 Individual areas can be created with the "a,number,keypoint1,keypoint2,keypoint3,keypoint4" command.
 All four keypoints are mandatory.
 The "number" means number of newly created area or it can be a number of existing one 
 (in this case the existing area is changed to new values). Please note that keypoints on area
 have to be ordered in COUNTER-CLOCKWISE direction.

 The "adel,node\_number" can be used to delete a area.



\subparagraph{Volumes}

 Individual volumes can be created with the "v,number,keypoint1,keypoint2,...,keypoint8" command.
 All four keypoints are mandatory.
 The "number" means number of newly created volume or it can be a number of existing one 
 (in this case the existing volume is changed to new values). Please note that keypoints on volume
 have to be ordered in COUNTER-CLOCKWISE direction.

 The "vdel,node\_number" can be used to delete a volume.



\subsubsection{Finite element mesh}

 The creationg of a finite element mesh is simple.
 Just use the "mesh" command and the nodes and
 the elements will be created. There are no
 settings or parameters needed. But please note
 that the uFEM can create regular meshes only.

 Entities not idented to be meshed must be unselected.



\subsubsection{Nodes}

 Individual nodes can be created through "n,identifier,x,y,z"
 command.

 Nodes can be copied with the "ngen,number,offset\_x,offset\_y,offset\_z" command.
 Note that the "number" parameter means number of newly created
 nodes (excluding the original ones). All existing nodes
 are copied, if only a subset of nodes is meant to be copied
 then these nodes must be selected first.

 The "ndel,node\_number" can be used to delete a node.



\subsubsection{Elements}

 Default properties of created element can be predefined
 by "edef,element\_type,real\_set,material" command.
 The parameters are a user-defined numbers of properties which
 were previously defined by "et", "rs" and "mat" commands.

 Individual elements can be created through "e,identifier,element\_type,real\_set,material".
 This command creates an element but the nodes must be defined by
 "e,identifier,node\_1,node\_2,..." command (all node identifier have to be given).

 Elements can be copied with the "engen,number,offset\_x,offset\_y,offset\_z" command.
 Note that the "number" parameter means number of newly created
 elements (excluding the original ones). All existing elements
 are copied, if only a subset of elements is meant to be copied
 then these nodes must be selected first. The command also creates new nodes
 if they are needed for newly created elements.

 The "edel,node\_number" can be used to delete an element.



\subsubsection{Boundary conditions}

 Individual boundary conditions can be created through "d,node,type,size"
 command. The "type" is something like "ux", "uy", "uz" (displacements),
 "rotx", "roty", "rotz" (rotations), "stiffx", "stiffy", "stiffz" (stiffnesses).

 Please note that current version of uFEM grafics can not show symbols for stiffnesses
 and they can only be listed by "dlist" command.

 Boundary conditions can be copied with the "dgen,number,offset\_x,offset\_y,offset\_z" command.
 Note that the "number" parameter means number of newly created
 boundary conditions (excluding the original ones). All existing boundary conditions
 are copied, if only a subset of boundary conditions is meant to be copied
 then these boundary conditions must be selected first.

 The "ddel,bc\_number" can be used to delete a boundary conditions with identifier "bc\_number".



\subsubsection{Loads}

 Individual loads on nodes can be created through "f,node,type,size"
 command. The "type" is something like "fx", "fy", "fz" (forces),
 "mx", "my", "mz" (moments - if applicable).

 Loads on nodes conditions can be copied with the "fgen,number,offset\_x,offset\_y,offset\_z" command.
 Note that the "number" parameter means number of newly created
 loads (excluding the original ones). All existing loads
 are copied, if only a subset of them is meant to be copied
 then these boundary conditions must be selected first.

 The "fdel,f\_number" can be used to delete a boundary conditions with identifier "f\_number".



\subsubsection{Data checking}

 Some possible problems in data can be checked. The command
 tries to find elements with undefined properties or nodes
 and empty materials. The command fails if a problem is
 found. The error message with desription of the problem is
 printed into Terminal window.

 In some cases (for empty real set 
 data, low number of loads or boundary conditons) only a
 warning is printed but the commands does not fail. It is
 because in some special cases are these settings correct.



\subsection{Solution}

 The solution process includes:


\begin{enumerate}
\item  export of data to a solver-compatible format: "export,fem,file\_name"
\item  calling of the solver: "system,fem -p -i file\_name -o res\_name"
\item  postprocessor start: "gpost,1"
\item  reading of results: "rread,,res\_name"
\end{enumerate}

 The described operations are done by the "solve" command. The path
 to the solver must be properly configured for this command.

 Please note that the "solve" command selects all data (including
 loads) for a solution, so it can not be used for multi-loadstep
 analysis. The "ssolve" command must be used instead.



\subsection{Postprocessing}

\subsubsection{Result listing}

\paragraph{Deformations}
\begin{itemize}
\item  "prdef" lists deformations in all selected nodes
\end{itemize}

\paragraph{Reactions}
\begin{itemize}
\item  "prrs" lists reactions in all selected nodes
\end{itemize}

\paragraph{Results on elements}
\begin{itemize}
\item  "pres,type1,type2,.." lists results in integration (or other) points on all selected elements. "typei" is something like "s\_x" or "e\_xy". 
\end{itemize}

\subsubsection{Result plotting}

\paragraph{Deformations}
\begin{itemize}
\item  "pldef" plots structure with deformed shape
\end{itemize}

\paragraph{Reactions}
\begin{itemize}
\item  "plrs" plots reactions in all selected nodes
\end{itemize}

\paragraph{Results on elements}
\begin{itemize}
\item  "ples,type1,type2,.." plots results in integration (or other) points on all selected elements. "typei" is something like "s\_x" or "e\_xy". 
\end{itemize}

\subsubsection{Element selection by results}
\begin{itemize}
\item  "esel,style,val,result\_type,value\_from,value\_to", where:\begin{itemize}
\item  "style" is the same style as is in other selections (s, r, u)
\item  "result type" is something like "s\_x", "e\_xy", "M\_x"
\end{itemize}
\end{itemize}

\subsection{Advanced tasks}

\subsubsection{Graphics view manipulations}

 The graphical image of model can be moved, rotated and resized
 with use of buttons at the right side of the graphics
 windows. It is also possible to manipulate it with commands:


\begin{itemize}
\item  moving: "move,direction"
\item  rotation along axis: "rot,axis"
\item  zooming: "zoom,zoom\_level" or "unzoom,zoom\_level"
\end{itemize}

 Please note that "zoom\_level" paramater must alvays
 be nonzero and positive.
 Alternatively, the mouse can be used for this task:


\begin{itemize}
\item  dragging with left mouse button pressed ... moves the model
\item  dragging up and down with center mouse button pressed ... zooms (up) and unzooms (down) the model
\item  dragging with right mouse button pressed ... rotetes the model
\end{itemize}

 Please note thatthe model is rotated along its X and Z
 axes (they are shown in bottom right corner of the
 graphics window).



\subsubsection{Multiple load steps}

\paragraph{TODO}

\subsubsection{Advanced data selections}

\paragraph{TODO}

\subsubsection{Variables}

\paragraph{TODO}

\subsubsection{Macros}

 It is possible to write a user scripts (macros) and then call them to
 provide additional functionality. The user scripts can be
 saved in text files and called through "input,the\_file"
 command.



\paragraph{Macro structure}
\begin{enumerate}
\item  "script,name\_of\_script" command
\item  commands, one command fer line
\item  "endscript" command
\end{enumerate}

\paragraph{Use of (already defined) macro}

 The macro can be called through
 "runscript,name\_of\_script" command.



\paragraph{Available flow controls}
\begin{itemize}
\item  "for" loop:
\end{itemize}

 The "for" loop must have this structure:


\begin{itemize}
\item  "for,ivariable,from\_value,to\_value" command
\item  commands (one per line), the "ivariable" can be  used here (in icludes the number of current loop)

\item  "endfor" command
\end{itemize}

 The "for" loops can be nested.


\begin{itemize}
\item  "if" command:
\end{itemize}

 This command can be used for conditional execution.
 It have to have this structure:


\begin{itemize}
\item  "if,value1,condition\_symbol,value2" command
\end{itemize}

 The "condition\_symbol" can be one of "=",
 "<", ">", "<=", ">=". The values can also be
 represented by variables.


\begin{itemize}
\item  commands (one per line)
\item  "endif" command
\end{itemize}

\paragraph{Evaluation of expressions}

 Mathematical expessions with variables can be avaluated
 through "eval" and "ieval" comamnds.



\subparagraph{Floating point expressions}

 The "eval,result\_variable,variable1,operation,variable2" can be
 used. The "operation" can be "+", "-", "*", "/".

 This command can not be used if the results is
 expected to be an integer value.



\subparagraph{Integer expressions}

 The "ieval,variable1,operation,variable2" can be
 used. The "operation" can be "+", "-", "*", "/".



\paragraph{Writing of variables to file}

 The data obtained diring a macro execution also can be
 saved with "writevar,file\_name,variable\_name" command.
 The "file\_name" represents a name of a file which will
 be  appended with the value of "variable\_name" variable.

 There are also other related commands:


\begin{itemize}
\item  "newfile,file\_name" ... creates a new file named "file\_name"; existing file is emptyed
\item  "writenl,file\_name" ... appends a "new line" symbol to a file named "file\_name" (it is usefull for a formated output)
\end{itemize}

\paragraph{TODO}

\subsubsection{Advanced configuration}

\paragraph{TODO}

\subsubsection{Saving of graphics}
\begin{itemize}
\item  "gl2ppm,file.ppm" saves current graphics image to file "file.ppm"
\item  "gl2ps,file.eps" saves current graphics image to file "file.eps" (Encapsulated PostScript format); it is very slow and this functionality is optional (it may not be enabled in your uFEM version)
\end{itemize}

\section{Hardware and operating systems support	}

\subsection{Solver}

 The solver can run on any platform that include an ANSI-compatible
 C compiler (the C89 is enough). However, some special features
 (a parallel processing) may require special libraries and 
 a POSIX-compliant operating environment. It is known that
 solver can run on Linux (32bit and 64bit PC, ARM and PowerPC
 platforms), Solaris (SPARC, i386), IRIX (MIPS), AIX (POWER),
 Mac OS X (PowerPC), Win32.



\subsection{Text-mode pre and postprocessor}

 This component has the same requirements as the solver has.



\subsection{Graphics pre and postprocessor}

 For a fully-functional program, these components are required:


\begin{itemize}
\item  Gtk+ 2.x  http://www.gtk.org
\item  GtkGlExt http://gtkglext.sf.net
\item  OpenGL http://www.opengl.org
\item  libTiff (optional - necessary only for outputs)
\item  libGl2PS (optional - necessary only for outputs)
\end{itemize}

 There is an option to build a limited version of program.
 The limited system needs the Gtk+ 1.x, GtkGlArea and the OpenGL
 libraries only but it has only a limited GUI funtionality. 	



\subsection{MS Windows issues}

\begin{enumerate}
 \item
 Only a fully functional version (but without libtiff and
 libgl2ps) has been tested. No noticeable issues so far, except
 unavailability of saving of graphics ouputs (no PPM nor EPS
 output can be used).
 The program has a strange (non-native) look here.
 \item Windows 7-10 issues: The Gtk+ version behaves strangely
 (doubleclick on a command tree crashes the program). There are
 several compilers which may or may not produce working code with
 different speed.
\end{enumerate}



\subsection{Mac OS 10.3 issues}

 Graphic selection are fully functional here but
 the selection frame is invisible. Also the X11 framework is
 required to run the uFEM. Only PowerPC-based Macs have been
 tested so far.



\subsection{Solaris 8, 9 issues}

 Reading of environment variables causes program exit
 (e. g. checking of UFEM\_CONFIG - it must not be used here).
 Solaris 10 is still untested. There are also color issues
 on SPARC platform (red image).



\section{Revisions and changes of this text}
\begin{itemize}
\item  Thu Apr 13 21:19:55 CEST 2017: fixes of tzping mistakes, port issues.
\item  Mon Jun 30 21:38:27 CEST 2008: fixes, one-dir loads, ssolve etc.
\item  Thu May 29 11:24:10 CEST 2008: elements, forces, disps., geom. ents.
\item  Wed May 28 22:54:53 CEST 2008: postprocessing, material types
\item  Mon Mar 24 20:43:10 CEST 2008: nodes, keypoints, help, selections
\item  Wed May 23 01:30:27 CEST 2007: "mat", "mp" commands added
\item  Wed Feb  7 23:29:25 CEST 2007: "rs" added, multistep finished
\item  Tue Jan  9 23:30:53 CEST 2007: fixes in solver descripion
\item  Tue Jan  1 09:10:00 CEST 2007: writing started
\end{itemize}

\end{document}
